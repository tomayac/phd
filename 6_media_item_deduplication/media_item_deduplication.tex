\chapter{Media Item Deduplication}
\label{sec:media-item-deduplication}

% the code below specifies where the figures are stored
\ifpdf
    \graphicspath{{6_media_item_deduplication/figures/PNG/}{6_media_item_deduplication/figures/PDF/}{6_media_item_deduplication/figures/}}
\else
    \graphicspath{{6_media_item_deduplication/figures/EPS/}{6_media_item_deduplication/figures/}}
\fi

\section{Introduction}

In the previous \autoref{cha:media-item-extraction}
in \autoref{sec:the-need-for-media-item-deduplication},
we have motivated the need for media item deduplication.
By clustering media items, we get a~higher level view on
a~media item cluster's overall performance on different networks.
As detailed in \autoref{sec:definition}, media items can be
photos or videos.
WordNet~\cite{fellbaum1998wordnet,miller1995wordnet} defines
the term \emph{duplicate} as
\textit{``a copy that corresponds to an original exactly''}.
the corresponding verb \emph{to duplicate} is defined as to
\textit{``make a~duplicate or duplicates of''}.
The derived term \emph{deduplication} in consequence refers to
the act of eliminating duplicate or redundant information.

\subsection{Definition of Exact Duplicate for Photos}

We define two media items of type photo as \emph{exact duplicate},
if their pixel contents are exactly the same.
This implies that by our definition, a~scaled version
of the same photo is \emph{not} an exact duplicate. 
Similarly, a~rotated version of a~photo is also \emph{not}
an exact duplicate. 
In contrast, two photo files with different file names,
or different Exchangeable image file
format\footnote{\url{http://www.cipa.jp/english/hyoujunka/kikaku/pdf/DC-008-2010_E.pdf},
accessed November 22, 2012}
(Exif) data, are considered exact duplicate,
if their pixel contents are exactly the same.
Exact duplicate photos typically occur if users share content 
from one social network on another, for example,
if one user posts a~photo on Instagram that then someone else
(or even the same user) posts on Facebook.

\subsection{Definition of Near-Duplicate for Photos}

We define two media items of type photo as \emph{near-duplicate},
if their pixel contents differ no more than a~given threshold.
Examples of near-duplicate photos are scaled versions
of the same photo, photos shot from a~slightly different angle,
rotated photos up to a~certain degree, \emph{etc.}
Near-duplicate photos typically occur if event attendants
stand close to each other and thus take photos
from a~similar standpoint.
Another scenario is a~user applying a~photo effect to a~photo
(like an Instagram filter) and then sharing both, the modified,
and the unmodified version.

\subsection{Definition of Exact Duplicate for Videos}

We define two media items of type video as \emph{exact duplicate},
if their pixel contents are frame by frame exactly the same.
In practice, we lower this condition and instead of every frame
only consider frames at shot boundaries.
We make \emph{no} requirements on the audio, \emph{i.e.},
a video in two different languages, however, that fulfills the 
pixel contents equality condition, is considered exact duplicate.
Typical scenarios where exact duplicate videos can occur is,
for example, two users sharing the same YouTube video
independently from each other.

\subsection{Definition of Near-Duplicate for Videos}

We define two media items of type video as \emph{near-duplicate},
if their pixel contents per frame differ no more
than a~given threshold.
In practice, we lower this condition and instead of every frame
only consider frames at shot boundaries.
Typical scenarios where near-duplicate videos can occur is through
logo or subtitle insertion, resizing, re-encoding,
or aspect-ration changes.
Note, we do not consider video subsegments near-duplicates.

\subsection{Special Case of Photo Contained in a~Video}

We define the special case of
\emph{a photo being contained in a~video} if the pixel contents
of a~photo media item differ no more than a~given threshold from
the pixel contents of any of the frames of a~video media item.
In practice, we lower this condition and instead of every frame
only consider frames at shot boundaries.
Typically, this phenomenon occurs if two event attendants
of the same event both cover the event from almost the same
standpoint, however, if the one attendant takes a~video,
while the other attendant takes a~photo.

\subsection{Motivation and Chapter Outline}

In this chapter, we will treat video
and photo deduplication separately. 
Our goal is to deduplicate media items on-the-fly
at the very moment they are extracted from social networks.
Due to this limitation, we cannot rely on any preprocessing
at all that state-of-the-art algorithms rely on.
This is why we dedicate a~whole section entirely to on-the-fly
shot boundary detection for online video,
which is by no means a~solved problem.
The contribution of our approach is that it is entirely Web-based
and on-the-fly, which introduces interesting new challenges
that traditional approaches to shot boundary detection
do not have to cope with.
Our Web-based approach abstracts away most of the low-level details
like the video codec, in favor of the high-level \texttt{<video>}
API, however, this also comes at a~cost.
A~major issue is the uncertain streaming speed,
where traditional approaches have immediate access
to the video file on disk.
An additional challenge is the unknown key frame distribution
of the target videos, which---together with streaming speed
issues---makes exact frame-wise video navigation impossible.
Our approaches to video and photo near-duplicate
and exact duplicate detection are founded
on a~tile-wise histogram-based pixel comparison algorithm.

\section{Video Shot Boundary Detection}
Video shot boundary detection is the processor-intensive task
of splitting a~video into continuous camera shots,
with hard or soft cuts as the boundaries.
In this section, we present a~browser-based, client-side, and
on-the-fly approach to this challenge
based on modern HTML5~\cite{berjon2012html5} Web APIs.
Once a~video has been split into shots,
shot-based video navigation gets enabled,
more fine-grained playing statistics can be created,
and finally, shot-based video comparison is possible.
\autoref{fig:screenshot} shows detected camera
shots for a~sample video.
The algorithm has been incorporated in a~browser extension
such that it can run transparently on a~major online video portal.

\begin{figure}
  \begin{center}
    \includegraphics[width=0.7\linewidth]{./stevejobs.png}
  \end{center}
  \caption{Camera shots for a~sample video on 
    a~major online video portal, detected on-the-fly via
    our shot boundary algorithm incorporated
    in a~browser extension.}
  \label{fig:screenshot}
\end{figure}

\subsection{Related Work} \label{sec:related-work}
Video fragments consist of shots, which are sequences of
consecutive frames from a~single viewpoint,
representing a~continuous action in time and space.
The topic of shot boundary detection has already been described
extensively in literature.
While some specific issues still remain
(notably gradual transitions and false positives
due to large movement or illumination changes),
the problem is considered resolved for many
cases~\cite{yuan2007shotboundary,hanjalic2002shotboundary}.
Below, we present an overview of several well-known categories of shot boundary detection techniques.

\emph{Pixel comparison
methods}~\cite{hampapur1994videosegmentation,
zhang1993videopartitioning} construct a~discontinuity metric
based on differences in color or intensity values
of corresponding pixels in successive frames.
This dependency on spatial location makes this technique
very sensitive to (even global) motion.
Various improvements have been suggested, such as prefiltering
frames~\cite{zhang1995videoparsing},
but pixel-by-pixel comparison methods proved inferior in the end
and have steered research towards other directions.

A~related method is
\emph{histogram analysis}~\cite{otoole1999shotboundary},
where changes in frame histograms are used
to justify shot boundaries.
Their insensitivity to spatial information
within a~frame makes histograms less prone to partial
and global movements in a~shot.

As a~compromise, a~third group of methods consists of
a~\emph{trade-off between the above two
techniques}~\cite{ahmed1999keyframe}.
Different histograms of several, non-overlapping blocks
are calculated for each frame,
thereby categorizing different regions of the frame
with their own color-based, space-invariant fingerprint.
The results are promising, while computational complexity
is kept to a~minimum, which is why we have chosen
a~variation of this approach for our own algorithm.

Other approaches to shot boundary detection include
the \emph{comparison of mean and standard deviations}
of frame intensities~\cite{lienhart1999comparison}.
Detection using other features such as
edges~\cite{zabih1995scenebreaks} and
motion~\cite{bouthemy1997shotchange} have also been proposed.
However, Gargi \emph{et~al.} have shown that
these more complex methods do not necessarily
outperform histogram-based approaches~\cite{gargi2000videoshot}.
A~detailed comparison can be found in
Yuan~\emph{et~al.}~\cite{yuan2007shotboundary}.

\subsection{Shot Boundary Detection Algorithm}
\label{sec:details-of-algo}

In this section, we discuss our shot boundary detection algorithm,
which falls in the category of histogram-based algorithms.
Since visually dissimilar video frames
can have similar global histograms,
instead we take local histograms into account. 
We therefore split video frames in freely configurable
rows and columns, \emph{i.e.}, lay a~grid of tiles over each frame.
The user interface, as can be seen in \autoref{fig:algorithm},
currently allows for anything from a~$\mathit{1} \times \mathit{1}$ 
grid to a~$\mathit{20} \times \mathit{20}$ grid.
For each step, we examine a~frame $\mathit{f}$ and its direct
predecessor frame $\mathit{f - 1}$.

Apart from the per-tile average histogram distance,
the frame distance function further considers
a~freely configurable number of \emph{most different} and
\emph{most similar} tiles.
This is driven by the observation that different parts
of a~video have different intensities of color changes,
dependent on the movements from frame to frame.
The idea is thus to boost the influence of movements in the frame
distance function, and to limit the influence of permanence.
In the debug view of our approach, as depicted in
\autoref{fig:algorithm}, blue boxes indicate movements,
while red boxes indicate permanence.
In the concrete example, Steve Jobs' head and shoulders move
as he talks, which can be clearly seen
at the blue boxes in the particular tiles.
Additional movements come from a~swaying flag on the left,
and a~plant on the right.
In contrast, the speaker desk, the white background,
and the upper part of his body remain static,
resulting in red boxes.
For this example, we use a~grid layout of
$\mathit{20} \times \mathit{20}$ tiles
($\mathit{nTiles} = \mathit{400}$),
and an empirically determined  number
$\mathit{tileLimit}$ of most different or similar tiles,
\emph{i.e.}, we treat one third of all tiles
as most different tiles, one third as normal tiles,
and one third as most similar tiles,
and apply boosting and limiting factors to the most different
and most similar tiles respectively.
We work with values of~$\mathit{1.1}$ for the
$\mathit{boostingFactor}$, which slightly increases
the impact of the most different tiles,
and $\mathit{0.9}$ for the $\mathit{limitingFactor}$,
which slightly decreases the impact of the most similar tiles.
The algorithm pseudo code can be seen in \autoref{code:algorithm}.

We define the average histogram distance between two frames
$\mathit{f}$ and $\mathit{f - 1}$ as $\mathit{avgHisto}_{f}$.
In a~first step, we have examined the histogram distance
data statistically, and observed that while
the overall average frame distance $\mathit{avgDist}_{f}$,
defined as $$\mathit{avgDist}_{f} =
\frac{1}{\mathit{nTiles}}\sum_{t=1}^{\mathit{nTiles}}
\mathit{avgHisto}_{f, t}$$ is very intuitive to human beings,
far more value lies in the standard deviation
$\mathit{stdDev}_{f}$, based on the definition of the overall
average frame distance $\mathit{avgDist}_{f}$
$$\mathit{stdDev}_{f} =
\sqrt{\frac{1}{\mathit{nTiles}}\sum_{t=1}^{\mathit{nTiles}}
(\mathit{avgHisto}_{f, t} - \mathit{avgDist}_{f})^{2}}$$
We use the standard deviation as a~value for the shot splitting
threshold~\cite{lienhart1999comparison}
to come to very accurate shot splitting results.
We found the boosting and limiting factors to have an overall
positive quality impact on more lively videos,
and a~negative quality impact on more monotone videos.
Best results can be achieved if,
after changing either the boosting or the limiting factors
for the most similar or different tiles,
the value of the shot splitting threshold is adapted
to the new resulting standard deviation.
The user interface optionally does this automatically.

\begin{figure}
  \begin{center}
    \includegraphics[width=1.0\linewidth]{./algorithm.png}
  \end{center}
  \caption{Debug view of the shot boundary detection process.
    Blue boxes highlight tiles with the most differences
    to the previous frame, red boxes those with most similarities.}
  \label{fig:algorithm}
\end{figure}

\begin{lstlisting}[caption=Pseudocode of shot boundary detection
  algorithm.,
  label=code:algorithm, float]
for frame in frames
  f = frame.index  
  for tile in tiles of frame      
    avgHisto[f][tile] = getTilewiseDiff()
 
  mostDiffTiles = getMostDiffTiles(avgHisto[f])
  mostSimTiles = getMostSimTiles(avgHisto[f])
 
  for tile in tiles of frame    
    factor = 1  
    if tile in mostDiffTiles
      factor = boostingFactor
    else if tile in mostSimTiles
      factor = limitingFactor
    avgHisto[f][tile] = avgHisto[f][tile] * factor
  avgDist[f] = avg(avgHisto[f])
\end{lstlisting}

\subsection{Implementation Details}
\label{sec:implementation}

The complete video analysis process happens fully
on the client side.
We use HTML5 JavaScript APIs of the \texttt{<video>} and
\texttt{<canvas>} tags.
In order to obtain a~video still frame
from the \texttt{<video>} tag at the current video position,
we use the \texttt{drawImage()} function of the 2D context of the
\texttt{<canvas>} tag,
which as its first parameter accepts a~video.
We then analyze the video frame's pixels tile-wise
and calculate the histograms.
In order to retrieve the tile-wise pixel data
from the 2D context of the \texttt{<canvas>},
we use the \texttt{getImageData()} function.
For processing speed reasons, we currently limit our approach to
a~resolution of one second, \emph{i.e.},
for each analysis step,
seek the video in $\mathit{1s}$ steps.
We then calculate the frame distances as outlined in
\autoref{sec:details-of-algo}.
For each frame, we can optionally generate an \texttt{<img>} tag
with a~base64-encoded data URI representation
of the video frame's data
that can serve for filmstrip representations of the video.

\subsection{Evaluation} \label{sec:evaluation}

Detecting shots on-the-fly in streaming video
comes with its very own challenges.
First, it is a~question of streaming speed.
Especially with high-definition (HD) video,
this can be very demanding.
We do not attach the analysis \texttt{<video>} tag
to the DOM tree~\cite{lehors2004dom} to save some CPU cycles,
however, the video still needs to be sought to each frame
in one second-steps and be processed.
Even on a~higher-end computer (our experiments ran on a~MacBook
Pro, Intel Core 2 Duo 2,66 GHz, 8 GB RAM),
the process of analyzing and displaying in parallel
a~$\mathit{1280} \times \mathit{720}$ HD video of media type
\emph{video/mp4; codecs="avc1.64001F, mp4a.40.2"}
caused an average CPU load of about 70\%.
The HTML5~\cite{berjon2012html5} specification states that
\textit{``when the playback rate is not exactly 1.0,
hardware, software, or format limitations can cause video frames
to be dropped''}.
In practice, this causes the analysis environment
to be far from optimal.
In our experiments we differentiated between false positives,
\emph{i.e.}, shot changes that were detected,
but not existent, and misses, \emph{i.e.},
shot changes that were existent,
but not detected.
Compared to a~set of videos with manually annotated shot changes,
our algorithm detected fewer false positives than misses.
The reasons were gradual transitions and shots
shorter than one second (below our detection resolution)
for misses, and large movements in several tiles
for false positives.
Overall, we reached an accuracy of about 86\%,
which is not optimal, but given the challenges
sufficient for our use case of
detecting near- or exact duplicate videos. 

\subsection{Optimization Potential}

Optimization potential lies in
improving the analysis speed by dynamically selecting
lower quality analysis video files,
given that videos are oftentimes available in several resolutions
(both LD and HD).
We will check in how far analysis results differ
for the various qualities.
Second, more advanced heuristics for the various user-definable
options in the analysis process are needed.
While there is no optimal configuration for all types of videos,
there are some key indicators that can help categorize videos
into classes and propose predefined known working settings
based on the standard deviation $\mathit{stdDev_{f}}$
and the overall average frame distance $\mathit{avgDist_{f}}$.
Both are dependent on the values of $\mathit{boostingFactor}$,
$\mathit{limitingFactor}$, $\mathit{rows}$, and $\mathit{columns}$. 
Interpreting our results so far, there is evidence
that low complexity settings are sufficient in most cases,
\emph{i.e.}, a~number of $\mathit{rows}$ and $\mathit{columns}$
higher than~$\mathit{2}$ does not necessarily
lead to more accurate shot boundary detection results.
The same applies to the number of to-be-considered most different
or similar tiles $\mathit{tileLimit}$.
We even had cases where not treating those tiles differently
at all, \emph{i.e.}, setting
$\mathit{boostingFactor} = \mathit{limitingFactor} = \mathit{1}$, 
led to better results.

\section{Photo Deduplication}

We determine the popularity of media items shared across
social networks.
This task involves the deduplication of extracted media items.
In the previous section, we have presented an algorithm
for on-the-fly shot boundary detection for video media items.
In this section, we will show how components of this algorithm
can be used to deduplicate images.
First, we look at related work for the task of photo deduplication.

\subsection{Related Work}

Work on ordinal measures that serve as a~general tool for
image matching was performed by Bhat \emph{et al.}
in~\cite{bhat1998imagecorrespondence}.
Chum \emph{et al.} have proposed a~near-duplicate image detection method using min-Hash and
term frequency--inverse document frequency (tf--idf)
weighting~\cite{chum2008nearduplicate}.
The proposed method uses a~visual vocabulary of vector quantized local feature descriptors based on
Scale Invariant Feature Transform (SIFT)~\cite{lowe1999sift}.
A~method for both photos and video has been proposed by Yang \emph{et al.}~\cite{yang2009nearduplicate}.

\subsection{Experiments}

\subsection{Evaluation}

\section{Video Deduplication}

\subsection{Related Work}

In addition to the before-mentioned  combined approach
for video and photo near-duplicate
detection~\cite{yang2009nearduplicate}, also more
specialized methods for video deduplication exist,
for example~\cite{min2011nearduplicatevideo,wu2009nearduplicate}.
A~survey of video deduplication methods has been conducted by
Lian \emph{et al.} in~\cite{lian2010survey}.

\subsection{Experiments}

\subsection{Evaluation}

\section{Conclusion}

\section*{Chapter Notes}
This chapter is partly based on the following publications:
\todo{Add publications}