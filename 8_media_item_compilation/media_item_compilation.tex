
% this file is called up by thesis.tex
% content in this file will be fed into the main document

%: ----------------------- introduction file header -----------------------
\chapter{Media Item Compilation}

% the code below specifies where the figures are stored
\ifpdf
    \graphicspath{{8_media_item_compilation/figures/PNG/}{8_media_item_compilation/figures/PDF/}{8_media_item_compilation/figures/}}
\else
    \graphicspath{{8_media_item_compilation/figures/EPS/}{8_media_item_compilation/figures/}}
\fi

% ----------------------------------------------------------------------
%: ------------------------------- content ----------------------------- 
% ----------------------------------------------------------------------

\section{Related Work}
\label{sec:related-work}

When unique media items have been collected, the remaining task is to summarize events by selecting the most relevant media fragments. Fabro and B\"osz\"orm\'enyi~\cite{Fabro:MMM12} detail the summarization and presentation of events from content retrieved from social media. Nowadays, many domain-specific methods already exhibit good accuracy, for example in the sports domain~\cite{Li1,Li2}. However, the challenge in this field is to find methods that are content-agnostic. Methods that exploit semantic information~(\emph{e.g.} \cite{Chen}) will likely provide high-quality results in the future, but today's most relevant summaries are produced by user interaction~\cite{Olsen}.


\section{Event Summarization}

\section{Visual Summarization}

\section{Audial Summarization}

\section{State of the Art}

\section{Evaluation}

\section{Conclusion}

\section*{Chapter Notes}
This chapter is partly based on the following publications:
\todo{Add publications}