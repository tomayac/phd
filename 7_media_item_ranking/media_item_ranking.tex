\chapter{Media Item Ranking}
\label{sec:media-item-ranking}

% the code below specifies where the figures are stored
\ifpdf
    \graphicspath{{7_media_item_ranking/figures/PNG/}{7_media_item_ranking/figures/PDF/}{7_media_item_ranking/figures/}}
\else
    \graphicspath{{7_media_item_ranking/figures/EPS/}{7_media_item_ranking/figures/}}
\fi

\section{Introduction}

In the previous chapter, we have motivated and shown methods
to deduplicate exact duplicate and near-duplicate media items.
The application screenshots that can be seen in \autoref{fig:topvsfashionshow}
and \autoref{fig:topgrammy} show the most intuitive ranking criterion
one can imagine: ranking by occurrence popularity.
The more often a~media item (or a~near-duplicate of it)
appears in any of the considered social networks, 
the higher it should be ranked.
Essentially, ranking by occurrence popularity (or media item cluster size)
disregards one of the most valuable features of social networks: 
the social aspects.
In consequence, in this chapter, we will introduce
further media item ranking criteria that,
together with media item cluster size,
will allow us to come up with more adequate social media item ranking mechanisms.

\section{Evaluating Subjective Data}

Evaluating subjective data, like \emph{the} correct ranking
for a~set of media items, is a~challenging task.
For different users, there may be different optimal settings.
A~common subjective evaluation technique
is the Mean~Opinion Score (MOS)~\cite{mos1998}.
Traditionally, MOS is used for conducting subjective evaluations
of telephony network transmission quality,
however, more recently, MOS has also found
wider usage in the multimedia community
for evaluating \emph{per se} subjective things
like perceived quality from the users' perspective. 
Therefore, a~set of standard, subjective tests are conducted,
where a~number of users rate the quality of test samples
with scores ranging from 1 (worst) to 5 (best).
The actual MOS is then the arithmetic mean of all individual scores.
Given a subjective evaluation criterion
like the correctness of a~ranking,
MOS provides a~meaningful way to judge the overall quality of our approach.

\section{Media Item Ranking Criteria}
In this section, we describe several feature categories that can serve to rank
media items retrieved from social networks. 
We assume (and are working on) media item extractors that,
given event-related search terms,
extract raw binary media items and associated microposts
from multiple social networks.

\noindent \textbf{Visual}
This category regards the contents of photos and videos.
We distinguish \emph{low-} and \emph{high-level} visual ranking criteria.
High-level criteria are, \emph{e.g.}, logo detection,
face recognition, and camera shot separation.
Low-level criteria are, \emph{e.g.}, size, resolution,
duration, geolocation, and time.
Via OCR, contained characters can be treated as textual features.

\noindent \textbf{Audial}
This category regards the audio track of videos.
\emph{High-level} ranking criteria are the presence or absence
of silence, music, speech, or a mixture thereof.
Similar to visual features before,
audial \emph{low-level} features are the average bit rate,
volume, possibly distorted areas, \emph{etc}.
Through audio-transcription, speech can be converted to a textual feature.

\noindent \textbf{Textual}
This category regards the microposts that accompany media items.
Typically, microposts provide a~description of media items.
Using named-entity disambiguation tools,
textual content can be linked to LOD cloud concepts~\cite{Facebook2011}.

\noindent \textbf{Social}
This category regards social network effects like shares, mentions,
view counts, expressions of (dis)likes, user diversity, \emph{etc}.
Prior work~\cite{Khrouf2012} allows us to not only examine these effects
on a~\emph{single} social network,
but in a~\emph{network-agnostic} way across multiple social networks.

\noindent \textbf{Aesthetic}
This category regards the desired outcome after the ranking, \emph{i.e.},
the media gallery that illustrates a given event and its atmosphere.
Studies exist for the aesthetics of
automatic photo book layout~\cite{Photo2011},
photo aesthetics \emph{per se}~\cite{Photo2012},
video and music playlist generation~\cite{YouTube2010,Playlist2006},
however media gallery composition requires mixing video
\emph{and} photo media items.

\section{Conclusion}

\section*{Chapter Notes}
This chapter is partly based on the following publications:
\todo{Add publications}