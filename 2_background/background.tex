
% this file is called up by thesis.tex
% content in this file will be fed into the main document

%: ----------------------- introduction file header -----------------------
\chapter{Background} \label{cha:background}

% the code below specifies where the figures are stored
\ifpdf
    \graphicspath{{2_background/figures/PNG/}{2_background/figures/PDF/}{2_background/figures/}}
\else
    \graphicspath{{2_background/figures/EPS/}{2_background/figures/}}
\fi

% ----------------------------------------------------------------------
%: ------------------------------- content ----------------------------- 
% ----------------------------------------------------------------------

The main contributions of this thesis are methods for the automatic generation of
user-customizable media galleries for the visual and audial summarization of events.
To provide context for the approaches proposed in the later parts of the thesis,
we start with an introduction of related background technologies.
The current \autoref{cha:background} covers the Semantic Web, Linked Data,
and the Resource Description Framework (RDF).
The following \autoref{cha:social-networks} will cover social networks
by first providing a~definition and classification of social networks,
and then introducing popular social networks and some of their core features.

The lexical database WordNet~\cite{Fellbaum1998} by the Cognitive Science Laboratory
of Princeton University defines\footnote{\url{http://wordnetweb.princeton.edu/perl/webwn?s=semantic}}
the term ``semantic'' as ``of or relating to meaning or the study of meaning''.
The same source defines\footnote{\url{http://wordnetweb.princeton.edu/perl/webwn?s=world+wide+web}}
the term ``Web'', which is a~common form for the complete term ``World Wide Web'' (or just ``WWW'') as
``computer network consisting of a~collection of internet sites that offer text and graphics and
sound and animation resources through the hypertext transfer protocol''.
Finally WordNet defines\footnote{\url{http://wordnetweb.princeton.edu/perl/webwn?s=meaning}}
the term ``meaning'' as ``the message that is intended or expressed or signified'', or
``the idea that is intended''.

The combined term ``Semantic Web'' has been coined by Sir Tim Berners-Lee,
the inventor of the World Wide Web and Director of the World Wide Web Consortium,
in a~May 2001 article co-published with James Hendler and Ora Lassila
in the Scientific American~\cite{BernersLee2001}.
Therein, the authors write: 

\begin{quotation}
``The Semantic Web will bring structure to the meaningful content of Web pages,
creating an environment where software agents roaming from page to page
can readily carry out sophisticated tasks for users. [\ldots]
The Semantic Web is not a~separate Web but an extension of the current one,
in which information is given well-defined meaning, better enabling computers and people
to work in cooperation.
The first steps in weaving the Semantic Web into the structure of the existing Web
are already under way.
In the near future, these developments will usher in significant new functionality
as machines become much better able to process and ``understand'' the data
that they merely display at present.''
\end{quotation}

\section{Semantic Web}

\section{Linked Data}

\section{Resource Description Framework}

\section{Conclusion}