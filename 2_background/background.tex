\chapter{Semantic Web Background Technologies}
\label{cha:background}

% the code below specifies where the figures are stored
\ifpdf
    \graphicspath{{2_background/figures/PNG/}{2_background/figures/PDF/}{2_background/figures/}}
\else
    \graphicspath{{2_background/figures/EPS/}{2_background/figures/}}
\fi

The main contributions of this thesis
are methods for the automatic generation of
user-customizable media galleries
for the visual and audial summarization of events.
To provide context for the proposed approaches
in the later parts of the thesis,
we start with two introductory chapters
related to Semantic Web background technologies
and social networks.
The current \autoref{cha:background} covers
the Semantic Web, Linked Data,
and the Resource Description Framework (RDF).
The following \autoref{cha:social-networks}
will cover social networks
by first providing a~definition
and classification of social networks,
and then introducing popular social networks
and some of their core features.

\section{The Web and Semantics}

Tim Berners Lee, inventor of the World-Wide Web (W3, WWW),
or simply, the \emph{Web}, \emph{et al.}
write in~\cite{bernerslee1994worldwideweb}:
\textit{``The World-Wide Web was developed
to be a~pool of human knowledge,
which would allow collaborators
in remote sites to share their ideas
and all aspects of a~common project''}.
Since the earliest days at CERN,
the European Organization for Nuclear Research
in Geneva, Switzerland,
the Web has scaled to a~truly global system
of interlinked hypertext documents
accessed via the Internet.

\emph{Semantics} is the study of meaning.
It focuses on the relation between
words, phrases, signs, and symbols,
and what they stand for, \emph{i.e.},
the actual object referred to by a~linguistic expression.
Michel Bréal can be counted as a~founder
of modern semantics with his 1897
\emph{Essai de sémantique}~\cite{breal1897essai}.
The \emph{Semantic Web} brings the two worlds---the
World-Wide Web and semantics---together.

\section{Semantic Web}

The lexical database
WordNet~\cite{fellbaum1998wordnet,miller1995wordnet}
by the Cognitive Science Laboratory
of Princeton University defines the term \emph{semantic}
as \emph{``of or relating to meaning or the study of meaning''}.
The same source defines the term \emph{Web},
which is a~common form for the complete term
\emph{World Wide Web} (or just \emph{WWW}) as
\emph{``computer network consisting of a~collection of internet sites that offer text and graphics and
sound and animation resources through the hypertext
transfer protocol''}.
Finally, WordNet defines the term \emph{meaning}
as \emph{``the message that is intended or expressed
or signified''}, or \emph{``the idea that is intended''}.

The combined term \emph{Semantic Web} was coined
by Sir Tim Berners-Lee,
in a~May 2001 article co-published with James Hendler
and Ora Lassila
in the Scientific American~\cite{bernerslee2001semanticweb}.
Therein, the authors write: 

\begin{quotation}
\textit{``The Semantic Web will bring structure to the meaningful
content of Web pages,
creating an environment where software agents
roaming from page to page
can readily carry out sophisticated tasks for users. [\ldots]
The Semantic Web is not a~separate Web
but an extension of the current one,
in which information is given well-defined meaning,
better enabling computers and people
to work in cooperation.
The first steps in weaving the Semantic Web
into the structure of the existing Web
are already under way.
In the near future, these developments
will usher in significant new functionality
as machines become much better able to process and \emph{understand} the data
that they merely display at present.''}
\end{quotation}

We currently experience a~fundamental shift
from the World Wide Web (WWW) to the Semantic Web,
a~shift from moving bits to moving bits with a~meaning.
This can have a~huge impact,
which might not be as drastic as Tim Berners-Lee describes
in his Scientific American article,
but which might introduce many small improvements,
like more accurate search results,
more intelligent price comparison services, \emph{etc.}
\autoref{fig:fundamental-shift} illustrates this idea.

\begin{figure}[h!]
\begin{center}
  \subfloat[Bits without meaning.]{
    \label{fig:fundamental-shift-1}
    \includegraphics[width=0.3\textwidth]{fundamental-shift-1.png}
   }                
  \subfloat[Bits with a~meaning.]{
    \label{fig:fundamental-shift-2}
    \includegraphics[width=0.468\textwidth]{fundamental-shift-2.png}
  }
  \caption{Fundamental shift from moving bits to moving bits with a~meaning.}
  \label{fig:fundamental-shift}  
\end{center}    
\end{figure}

\subsection{The Non-Semantic Web} \label{sec:non-semantic-web}

To differentiate the Semantic Web from the non-semantic Web,
it helps to step back one step and
see why the non-semantic Web is not semantic.
The Web is a~system of interlinked hypertext documents
accessed through the Internet.
These documents are typically marked up in
the Hypertext Markup Language (HTML),
a~language that defines a~syntax
understandable to user agents like Web browsers,
however, not one that provides meaning
beyond the level of text layout.
This means that an HTML snippet like
\begin{verbatim}
<h1>The Catcher in the Rye</h1>
<h2>J. D. Salinger</h2>
\end{verbatim}
reveals that \emph{The Catcher in the Rye}
is a~level one header element and
that \emph{J. D. Salinger} a~level two header element,
but to a~machine it is not evident that the prior
is the title of a~book,
and that the latter is (i) an author, and (ii)
the author of \emph{The Catcher in the Rye}.

\subsection{Structured Data on the Web}

A~very first step to add semantics to the Web
is using tabular data.
\autoref{tab:sample-table-structured-data} shows an example
for such tabular data.
For human beings (interested in sports),
the meaning of the columns in
\autoref{tab:sample-table-structured-data} is clear:

\begin{itemize}
  \item P = matches \textbf{P}layed
  \item W = \textbf{W}on 
  \item D = \textbf{D}rew
  \item L = \textbf{L}oss
  \item F = Goals \textbf{F}or
  \item A = Goals \textbf{A}gainst
  \item Pts = \textbf{P}oin\textbf{ts}
\end{itemize}

\begin{table}[b!]
  \begin{center}
    \begin{tabular}{l*{6}{c}r}
Team              & P & W & D & L & F  & A & Pts \\
\hline
Manchester United & 6 & 4 & 0 & 2 & 10 & 5 & 12  \\
Celtic            & 6 & 3 & 0 & 3 &  8 & 9 &  9  \\
Benfica           & 6 & 2 & 1 & 3 &  7 & 8 &  7  \\
FC Copenhagen     & 6 & 2 & 1 & 2 &  5 & 8 &  7  \\
    \end{tabular}
    \caption{Sample table with structured data for sports results.}
    \label{tab:sample-table-structured-data}
  \end{center}
\end{table}

The problem, however, is for machines to understand
the structure of the table.
Let us imagine one wanted to automate the task
of retrieving sports results from a~Web page with tabular data.
While it is a~straightforward job to implement
a~scraper bot that searches for column titles
like ``P'', ``W'', ``D'', \emph{etc.},
it would require the same work over and over again
for a~different language,
for example, German, where the termns would be:
``Sp.'', ``g.'', ``u.'', ``v.'', ``Tore'', ``Pkte.''.
A~German-speaking reader might have noticed
that the exemplary German system listed here
does not differentiate between \emph{goals for}
and \emph{goals against}, but only has a~list of \emph{Goals}.
Tiny differences like this make the scraping approach brittle.
If data providers were to use unique column identifiers like
Unique Resource Identifiers (URIs),
the problem would be easier.
In the concrete example for English and German,
rather than using ``D'' (Drew) and ``u.'' (unentschieden),
which both mean that the result was a~tie,
the machine-readable column name could be identified by the URI
\url{http://dbpedia.org/page/Tie_(draw)}.
In the next section, we therefore introduce
the structured knowledge base DBpedia~\cite{auer2007dbpedia}.

\subsection{The Structured Knowledge Base DBpedia}

An often reoccurring (however not binding) pattern
in the Semantic Web world
is the use of DBpedia~\cite{auer2007dbpedia}
as a~hub for identifying concepts by URIs.
DBpedia is a~Semantic Web knowledge base
with the objective of automatically extracting
structured data from the human-generated information
from the online encyclopedia
Wikipedia~\footnote{\url{http://en.wikipedia.org/wiki/Main_Page}}.
This structured information is then made available
on the World-Wide Web in many formats,
for example, in JSON~\cite{crockford2006json}
and many RDF~\cite{klyne2004rdf} serializations.
DBpedia allows for querying relationships
and properties associated with Wikipedia resources,
including links to other related datasets.
As outlined before, the concept of a~tie draw
in the sense of sports could thus be uniquely identified
by the DBpedia URI \url{http://dbpedia.org/page/Tie_(draw)},
free of all ambiguity.
Similar knowledge bases are among others
Freebase~\cite{markoff2007freebase},
YAGO~\cite{suchanek2007yago}, and CYC~\cite{lenat1995cyc}.

\subsection{Semantics in HTML Versions 4.01 and 5}

As outlined in \autoref{sec:non-semantic-web},
HTML versions 4.01~\cite{raggett1999html}
and 5~\cite{berjon2012html5}
contain a~basic level of semantics.
The main focus, however, is on the separation of semantics
from presentation.
For example the \texttt{<b>} and the \texttt{<strong>} tags
both have the same visual effect:
they make the node value appear in a~bold face \textbf{like so}.
Visually, there is no way to differentiate between the two,
however, semantically the difference exists and is well-defined:
\texttt{<strong>} should be used when
one wants to give special emphasis on something,
screen readers will typically read out such text
with a~more emphasized voice.
In contrast \texttt{<b>} should be used
if only visually one wants to create a~bold face look.
In the following, we present a~list of semantic HTML tags
and attributes and their meaning.

\paragraph{Semantic HTML4.01 Tags}

\begin{itemize}
  \item \texttt{<abbr>} specifies an abbreviation,
        \texttt{<acronym>} specifies an acronym.
  \item \texttt{<h1>--<h6>} specify level 1--6 headers,
        \texttt{<caption>} specifies a~caption for a~table.
  \item \texttt{<blockquote>} specifies a~block-level quotation
        (a~source in form of a~URI may be specified via the
        \texttt{@cite} attribute),
        \texttt{<cite>} specifies a~citation.
  \item \texttt{<dl>} specifies a~definition list, \texttt{<dt>}
        specifies a~definition term in a~definition list,
        \texttt{<dd>} specifies the definition of a~term
        in a~definition list.
  \item \texttt{<em>} specifies an emphasis, \texttt{<strong>}
        specifies a~strong emphasis.
  \item \texttt{<code>} specifies a~code snippet, \texttt{<dfn>}
        specifies an inline definition of a~single term,
        \texttt{<address>} specifies contact information
        for the document author, \texttt{<legend>} specifies
        a~legend for \texttt{<fieldset>} containers
        for adding structure to forms,
        \texttt{<samp>} specifies sample output
        from a~script or program.
\end{itemize}

\paragraph{Semantic HTML5 Tags}

\begin{itemize}
  \item \texttt{<article>} specifies an independent item
        section of content,
        \texttt{<aside>} specifies a~section of a~page that
        consists of content that is tangentially related
        to the content around the \texttt{<aside>} element,
        and which could be considered separate from
        that content, \texttt{<header>} specifies a~group of
        introductory or navigational aids,
        \texttt{<footer>} specifies a~footer for its nearest
        ancestor sectioning content or sectioning root element,
        \texttt{<nav>} specifies a~section with navigation links.
  \item \texttt{<figure>} specifies some flow content,
        \texttt{<mark>} specifies a~run of text in one document
        marked or highlighted for reference purposes
        due to its relevance in another context,
        \texttt{<meter>} specifies a~scalar measurement within
        a~known range, or a~fractional value.
  \item \texttt{<audio>} specifies a~sound or an audio stream,
        \texttt{<video>} specifies a~video or movie.
  \item \texttt{<progress>} specifies the completion progress
        of a~task, \texttt{<time>} specifies either a~time
        on a~24 hour clock, or a~precise date in the calendar
        (optionally with a~time and a~time-zone offset),
        \texttt{<command>} specifies a~command that the user
        can invoke.
  \item \texttt{<details>} specifies a~disclosure widget
        from which the user can obtain additional information or
        controls, \texttt{<datalist>} specifies the list that
        represent predefined options for input elements.
  \item \texttt{<keygen>} specifies a~key pair generator control,
        \texttt{<output>} specifies the result of a~calculation,
        \texttt{<ruby>} allows one or more spans of phrasing
        content to be marked with ruby annotations.
\end{itemize}

\paragraph{HTML5 Input Attributes}

\begin{itemize}
  \item \texttt{@datetime} specifies a~control for setting the
        element’s value to a~string representing a~global date
        and time (with timezone information).
  \item \texttt{@datetime-local} specifies a~control for setting
        the element’s value to a~string representing a~local date
        and time (with no timezone information).
  \item \texttt{@date} specifies a~control for setting the 
        element’s value to a~string representing a~date,
        \texttt{@month} specifies a~control for setting the
        element’s value to a~string representing a~month,
        \texttt{@week} specifies a~control for setting the
        element’s value to a~string representing a~week.
  \item \texttt{@time} specifies a~control for setting the
        element’s value to a~string representing a~time
        (with no timezone information).
  \item \texttt{@number} specifies a~control for setting the
        element’s value to a~string representing a~number.
  \item \texttt{@range} represents an imprecise control
        for setting the element’s value to a~string
        representing a~number.
  \item \texttt{@email} specifies a~control for editing
        a~list of email addresses given in the element’s value.
  \item \texttt{@url} specifies a~control for editing an
        absolute URL given in the element’s value.
  \item \texttt{@search} specifies a~one-line plain-text
        edit control for entering one or more search terms.
  \item \texttt{@color} specifies a~color-well control for
        setting the element’s value to a~string representing
        a~simple color.
\end{itemize}

\subsection{Structured Data Beyond Pure HTML}

In this subsection, we describe how structured data
can be included in HTML documents
by either overloading existing HTML attributes,
or by adding new ones.

\subsubsection{Microformats}

Microformats~\cite{celik2006microformats} are a~set of
open data mark-up formats developed
and defined by the Microformats
community\footnote{\url{http://microformats.org/discuss}}.
Microformats are not an official standard,
but rather a~widely adopted grass-roots-driven movement
with origins in the blogging scene.
It is to be noted that Microformats do not require a~new language,
but reuse building blocks from widely adopted standards
such as the \texttt{@class}, \texttt{@rel}, and \texttt{@title}
attributes in HTML.
Their main design goal is to focus first on humans,
then on machines.
A~concrete example of Microformat mark-up in HTML
can be seen in \autoref{code:microformats}.
There are currently nine stable
Microformats\footnote{\url{http://microformats.org/wiki/Main_Page\#Specifications}},
as listed below:

\begin{itemize}
  \item \texttt{hCalendar} is a~distributed calendaring and
        events format, using a~1:1 representation of the standard
        \texttt{iCalendar} format
        (RFC~2445,~\cite{dawson1998icalendar}).
  \item \texttt{hCard} is a~format for representing people,
        companies, organizations, and places, using a~1:1
        representation of the standard \texttt{vCard} format
        (RFC~2426,~\cite{dawson1998vcard}).
  \item \texttt{rel-license} is a~format for indicating content
        licenses, which is embeddable in
        HTML~\cite{raggett1999html} or
        XHTML~\cite{pemberton2000xhtml},
        Atom~\cite{nottingham2005atom},
        RSS~\cite{cadenhead2006rss},
        and arbitrary XML~\cite{bray2008xml}.
  \item \texttt{rel-nofollow} is a~format for hyperlinks
        indicating that the destination of that hyperlink should
        not be afforded any additional weight or ranking by user
        agents such as search engines, which perform link
        analysis upon Web pages.
  \item \texttt{rel-tag} is a~format for hyperlinks indicating
        that the destination of that hyperlink is an
        author-designated keyword for the current page.
  \item \texttt{VoteLinks} is a~format for adding the idea of
        agreement, abstention or indifference, and disagreement
        to hyperlinks.
  \item \texttt{XFN} is a~format for representing human
        relationships using hyperlinks, which enables Web authors
        to indicate their relationships to people.
  \item \texttt{XMDP} is a~format for defining metadata profile
        documents (XHTML Meta Data Profile), which enables Web
        authors to well-define custom meta tags.
  \item \texttt{XOXO} is a~format for defining a~new
        XHTML~\cite{pemberton2000xhtml}
        document type for subsetting and extending XHTML,
        which serves as the basis for XHTML-friendly outlines for
        processing by XML engines, and for easy interactive
        rendering by browsers.
\end{itemize}

\begin{lstlisting}[caption={
   [Sample code snippet with embedded hCard Microformat mark-up.]
   {Sample code snippet with embedded \texttt{hCard} Microformat
    mark-up. Source: \url{http://microformats.org/wiki/hcard}.}
  },
  label={code:microformats}]
<div class="vcard">
  <a class="fn org url" href="http://www.commerce.net/">CommerceNet</a>
  <div class="adr">
    <span class="type">Work</span>:
    <div class="street-address">169 University Avenue</div>
    <span class="locality">Palo Alto</span>,  
    <abbr class="region" title="California">CA</abbr>&nbsp;&nbsp;
    <span class="postal-code">94301</span>
    <div class="country-name">USA</div>
  </div>
  <div class="tel">
   <span class="type">Work</span> +1-650-289-4040
  </div>
  <div class="tel">
    <span class="type">Fax</span> +1-650-289-4041
  </div>
  <div>Email: 
   <span class="email">info@commerce.net</span>
  </div>
</div>
\end{lstlisting}

\subsubsection{Microdata}

Microdata~\cite{hickson2012microdata} defines a~way to annotate 
content (or items) with specific machine-readable labels,
for example, to allow scripts to provide services that are
customized to a~website.
Microdata allows for nested groups of name-value pairs
to be added to documents,
in parallel with the existing content.
The Microdata specification introduces
a~set of new attributes to HTML:

\begin{itemize}
  \item \texttt{@itemscope} creates an item (or thing) and
        indicates that descendants of this element contain
        information about it. This attribute precedes the
        \texttt{@itemtype} attribute in the HTML element's tag.
  \item \texttt{@itemtype} a~valid URL of a~vocabulary that
        describes the item and its properties context.
  \item \texttt{itemid} indicates a~unique identifier
        of the item in the vocabulary.
  \item \texttt{@itemprop} indicates that its containing tag
        holds the value of the specified item property.
        The properties name and value context are described by
        the items vocabulary. Properties values usually
        consist of string values,
        but can also use URLs using the \texttt{<a>} tag
        and its \texttt{@href} attribute,
        the \texttt{<img>} tag and its \texttt{@src} attribute,
        or other tags that link to or embed external resources.
  \item \texttt{@itemref} properties that are not descendants of
        the element with the \texttt{@itemscope} attribute
        can be associated with the item using this attribute.
        Provides a~list of elements to web crawlers to find
        additional property values of the item
        elsewhere in the document.
\end{itemize}

An example of Microdata in HTML can be seen in \autoref{code:microdata}.

\begin{lstlisting}[caption={
  [Sample code snippet with Microdata mark-up.]
  {Sample code snippet with Microdata mark-up.
   Source: \url{http://www.w3.org/TR/microdata/}.}},
  label={code:microdata}]
<div itemscope>
  <p>My name is <span itemprop="name">Neil</span>.</p>
  <p>
    My band is called
    <span itemprop="band">Four Parts Water</span>.
  </p>
  <p>I am <span itemprop="nationality">British</span>.</p>
</div>
\end{lstlisting}

\section{Resource Description Framework (RDF)} \label{sec:rdf}

The Resource Description Framework (RDF)~\cite{klyne2004rdf}
defines a~set of W3C standards for the formal description of
resources that are identified by URIs.
RDF is a~core component of the Semantic Web.
Initially, it was designed to describe metadata
on the World-Wide Web (WWW) such as authors,
copyrights, \emph{etc.} of documents, however,
applying a~definition of the term \emph{resource}
beyond the WWW context,
RDF is now also used to describe metadata
of any URI-identifiable entity.

\subsection{Triples as a~Data Structure}

As outlined before, one of the main purposes of the Semantic Web
is to give information a~well-defined meaning.
Using an example from Tim Berners-Lee’s
article~\cite{bernerslee2001semanticweb},
meaning can be to differentiate between the concepts of
a~shipping and a~billing address,
or the concept of an address in the sense of
delivering a~formal spoken communication to an audience.
In order to assure the differences in meaning,
things are identified by a~Unique Resource Identifier (URI).
The majority of the data processed by machines
can be described by elementary sentences like
\emph{A~cat is a~mammal},
\emph{Thomas Steiner is the author of this document},
or \emph{Prince William is married to Kate Middleton}.
Each of these sentences has a~subject (\emph{A~cat}),
a~predicate (\emph{is a}), and an object (\emph{mammal}).
Every subject, predicate, and object can be identified by a~URI.
This concept is very powerful,
as it allows to express the same concept,
represented by a~URI (like, for example, mammal by
\url{http://dbpedia.org/resource/Mammal})
with different terms in different languages
(like, for example, Säugetier, mammal, or nisäkkäät).
Everyone can extend the set of concepts,
simply by creating a~URI on the Web.
This form of knowledge representation
is used by the Resource Description Framework.

\subsection{Important RDF Serialization Syntaxes}

The knowledge represented in the RDF triple data structure
needs to be serialized in order to be stored
or transmitted over the Internet.
Several serialization formats exist,
each of which with its particular advantages
and disadvantages, mostly around readability for human beings
and parsability for machines.
Most people prefer the Turtle~\cite{prudhommeaux2011turtle}
format for its readability,
whereas for machines N-Triples~\cite{grant2004ntriples}
is the easiest to parse.

\subsubsection{RDF Sample Graph}
In the following, we will illustrate the various
RDF serialization formats
with an RDF sample graph, which is introduced here.
It contains data about a~fictive FOAF~\cite{brickley2010foaf}
person named \emph{John X. Foobar},
and an email address with an SHA1 checksum of
\emph{cef817456278b70cee8e5a1611539\-ef9d928810e}.
The actual email address is obscured to avoid spam emails.
\autoref{fig:sample-rdf-graph} shows the graphical representation
of this sample graph.

\begin{figure}[h!]
\begin{center}
  \includegraphics[width=\linewidth]{sample-rdf-graph.png} 
  \caption{A~sample RDF graph visualized.}
  \label{fig:sample-rdf-graph}
  \end{center}  
\end{figure}

\subsubsection{The Notation3 Syntax} \label{sec:notation3}

Notation3~\cite{bernerslee2011notation3} was introduced
by Tim Berners-Lee.
Notation3 has some features that go beyond
the pure expressiveness of RDF, like rules.
Its media type is \texttt{text/n3;\-charset=utf-8},
the recommended file extension is \texttt{.n3},
the encoding is always UTF-8.
\autoref{code:notation3-syntax} shows the previously
introduced sample graph serialized in Notation3.

\begin{lstlisting}[caption={A~sample graph in Notation3 syntax.},
  label={code:notation3-syntax}]
@prefix foaf: <http://xmlns.com/foaf/0.1/> .

_:node15urahancx74223 a foaf:Person ;
  foaf:name "John X. Foobar" ;
  foaf:mbox_sha1sum "cef817456278b70cee8e5a1611539ef9d928810e" .
\end{lstlisting}

\subsubsection{The Turtle Syntax} \label{sec:turtle}

Turtle~\cite{prudhommeaux2011turtle},
or the Terse RDF Triple Language, was defined by Dave Beckett.
It is a~superset of N-Triples (see \autoref{sec:n-triples}) and
a~subset of Notation3 (see \autoref{sec:notation3}).
It has reached a~\emph{de facto} standard status,
with the RDF Working Group publishing the new Turtle specification
as a~Last Call Working Draft on July 10, 2012.
Its media type is \texttt{text/turtle}
(the sometimes still observable media type
\texttt{application/x-turtle} is deprecated),
the recommended file extension is \texttt{.ttl},
the encoding is UTF-8.
\autoref{code:turtle-syntax} shows the previously
introduced sample graph serialized in Turtle.

\begin{lstlisting}[caption={A~sample graph in Turtle syntax,
  the syntax is equivalent to \autoref{code:notation3-syntax}.},
  label={code:turtle-syntax}]
@prefix foaf: <http://xmlns.com/foaf/0.1/> .

_:node15urahancx74223 a foaf:Person ;
  foaf:name "John X. Foobar" ;
  foaf:mbox_sha1sum "cef817456278b70cee8e5a1611539ef9d928810e" .
\end{lstlisting}

\subsubsection{The N-Triples Syntax} \label{sec:n-triples}

The N-Triples~\cite{grant2004ntriples} syntax was primarily
developed by Dave Beckett and Art Barstow.
N-Triples is a~subset of Turtle (see \autoref{sec:turtle}),
which in turn is a~subset of Notation3
(see \autoref{sec:notation3}).
There are very few variations to express a~graph in N-Triples,
which makes it an ideal syntax for testing purposes, however,
as it is missing some shortcuts of Turtle, it is quite verbose.
Its media type is \texttt{text/plain}, the recommended file extension is \texttt{.nt},
and the encoding is 7-bit US-ASCII
(and explicitly \emph{not} UTF-8).
\autoref{code:ntriples-syntax} shows the previously
introduced sample graph serialized in N-Triples.

\begin{lstlisting}[caption={A~sample graph in N-Triples syntax.},
  label={code:ntriples-syntax}]
_:1 <http://www.w3.org/1999/02/22-rdf-syntax-ns#type>
    <http://xmlns.com/foaf/0.1/Person> .
_:1 <http://xmlns.com/foaf/0.1/name>
    "John X. Foobar" .
_:1 <http://xmlns.com/foaf/0.1/mbox_sha1sum>
    "cef817456278b70cee8e5a1611539ef9d928810e" .
\end{lstlisting}

\subsubsection{The RDF/XML Syntax}

RDF/XML~\cite{beckett2004rdfxml} was introduced by
the W3C as the first RDF serialization syntax.
Albeit more human-friendly serialization formats
such as Turtle~\cite{prudhommeaux2011turtle}
gain more and more traction,
RDF/XML is still very wide-spread.
Its media type is \texttt{application/rdf+xml},
the recommended file extension is \texttt{.rdf},
the encoding is UTF-8.
\autoref{code:rdfxml-syntax} shows the previously
introduced sample graph serialized in RDF/XML.

\begin{lstlisting}[caption={A~sample graph in RDF/XML syntax.},
  label={code:rdfxml-syntax}]
<?xml version="1.0" encoding="UTF-8"?>
<rdf:RDF
    xmlns:foaf="http://xmlns.com/foaf/0.1/"
    xmlns:rdf="http://www.w3.org/1999/02/22-rdf-syntax-ns#">
  <rdf:Description rdf:nodeID="node15urahancx74224">
    <rdf:type rdf:resource="http://xmlns.com/foaf/0.1/Person"/>
    <foaf:name>John X. Foobar</foaf:name>
    <foaf:mbox_sha1sum>
      cef817456278b70cee8e5a1611539ef9d928810e
    </foaf:mbox_sha1sum>
  </rdf:Description>
</rdf:RDF>
\end{lstlisting}

\subsubsection{The RDFa Syntax}

RDFa~\cite{adida2012rdfa} has a~special role
in that it is a~specification for attributes
to express structured data in XHTML~\cite{pemberton2000xhtml},
but also in HTML4 and HTML5~\cite{sporny2012htmlrdfa}.
It uses the rendered hypertext content of (X)HTML
for the RDFa markup,
so that data publishers can use the same document
for human- and machine-readable content.
The contained RDF triples can be extracted with distillers,
in consequence RDFa can be considered
as another serialization syntax for RDF,
with the same expressive power as
RDF/XML~\cite{beckett2004rdfxml},
Turtle~\cite prudhommeaux2011turtle{}, \emph{etc.}
Its media type is \texttt{application/xhtml+xml},
the recommended file extension is \texttt{.html}.
RDFa shares some design goals
with Microformats~\cite{celik2006microformats}.
Where Microformats specify both a~syntax
for embedding structured data into HTML
\emph{and} a~vocabulary of specific terms for each Microformat,
RDFa in contrast \emph{only} specifies a~syntax.
RDFa relies on independent external specifications of vocabularies.
The essence of RDFa is a~set of attributes
that contain metadata about things,
and that can be embedded in mark-up languages,
for example in XHTML or HTML.
The concrete attributes are as follows.

\begin{itemize}
  \item \texttt{@about} and \texttt{@src} a~URI or
        CURIE (compact URI) that specifies the resource
        the metadata is about.
  \item \texttt{@rel} specifies a~relationship with
        another resource.
  \item \texttt{@href} and \texttt{@resource} specify
        the partner resource.
  \item \texttt{@property} specify a~property for
        the content of an element.
  \item \texttt{@content} optional attribute that overrides
        the content of the element when using
        the property attribute.
  \item \texttt{@datatype} optional attribute that
        specifies the datatype of text specified
        for use with the property attribute.
  \item \texttt{@typeof} optional attribute that specifies
        the RDF type(s) of the subject (the resource
        that the metadata is about).
\end{itemize}

\autoref{code:rdfa-syntax} shows the previously introduced
sample graph serialized in RDFa.

\begin{lstlisting}[caption={A~sample graph in RDFa syntax.},
  label={code:rdfa-syntax}]
<div about="_:1" typeof="http://xmlns.com/foaf/0.1/Person">           
  <span property="http://xmlns.com/foaf/0.1/mbox_sha1sum">
    cef817456278b70cee8e5a1611539ef9d928810e
  </span> 
  <span property="http://xmlns.com/foaf/0.1/name">
    John X. Foobar
  </span>
</div> 
\end{lstlisting}

\section{SPARQL: Semantic Web Query Language}

SPARQL is a~recursive acronym that stands for
SPARQL Protocol and RDF Query Language.
The SPARQL specification~\cite{prudhommeaux2008sparql}
defines the syntax and semantics
of the SPARQL query language for RDF.
RDF~\cite{klyne2004rdf} is a~directed, labeled
graph data format for representing information on the Web.
SPARQL can be used to express queries
across diverse data sources,
whether the data is stored natively as RDF,
or viewed as RDF via middleware.
SPARQL allows for querying required
and optional graph patterns
along with their conjunctions and disjunctions.
SPARQL also supports extensible value testing
and constraining queries by source RDF graph.
The results of SPARQL queries can be result sets or RDF graphs.
SPARQL became an official W3C Recommendation in 2008.
It was standardized by the RDF Data Access Working Group (DAWG).

\subsection{The Vision of the Web as a~Giant, Single Database}

The Web as we know it today is a~\emph{network of documents},
interconnected by hyperlinks that everyone can participate in
by placing links to existing documents.
The vision of the Semantic Web, however,
is a~\emph{network of facts} about entities,
interconnected by means of graphs of data.
Where the Web of today is a~graph of documents,
the Semantic Web is envisioned to be a~huge global graph,
formed by many individual graphs.
If one party publishes facts about an entity,
and a~different party publishes different facts
about the same entity,
then the overall knowledge about that entity
is represented in a~decentralized way,
accessible to all, and open for everyone to enrich.
This obviously requires strong globally unique identifiers,
or at least ways to map one identifier to another.

Given the (visionary) huge global graph, a~fictive
SPARQL query like the one in \autoref{code:sparql}
could be used to get results from the graph,
like the email addresses of every person in the world.

\begin{lstlisting}[caption={[SPARQL query returning the names and
  email addresses of every person in the world.]
  {SPARQL query returning the names and email addresses of every
  person in the world.
  Source: \url{http://en.wikipedia.org/wiki/SPARQL\#Benefits}.}},
  label={code:sparql}]
PREFIX foaf: <http://xmlns.com/foaf/0.1/>
SELECT ?name ?email
WHERE {
  ?person a foaf:Person.
  ?person foaf:name ?name.
  ?person foaf:mbox ?email.
}
\end{lstlisting}

This query selects the names and email addresses from all persons
who have facts about them in the global graph.
The query starts with a~prefix definition,
and then constrains the results
to be of type \texttt{foaf:Person}~\cite{brickley2010foaf},
whose name and email address are the values
of the triples with the predicates \texttt{foaf:name}
and \texttt{foaf:mbox} respectively.
While this sounds like a~very powerful idea,
for performance reasons in practice SPARQL endpoints
like the DBpedia SPARQL
endpoint\footnote{\url{http://dbpedia.org/sparql}}
typically only allow for querying a~local graph.

\subsection{Different SPARQL Query Variations}

The SPARQL Query Language currently specifies four different query variants, which we list in the following. 

\subsubsection{SELECT}

The \texttt{SELECT} query variant is used to extract
raw values from a~SPARQL endpoint.
The results are returned in a~table format.
A~sample query is given in \autoref{code:sparql}.

\subsubsection{DESCRIBE}

The \texttt{DESCRIBE} query variant is used to extract
an RDF graph from the SPARQL endpoint,
the contents of which is left to the endpoint to decide
based on what the maintainer deems as useful information.
An example query is given below:\\

\texttt{DESCRIBE <http://example.org/sparql>}

\subsubsection{ASK}

The \texttt{ASK} query variant is used to provide
a~simple \texttt{true} or \texttt{false} result
for a~query on a~SPARQL endpoint.
No information is returned about the possible query solutions,
just whether or not a~solution exists.
An example query with a~sample response is given below:\\

\noindent Given the RDF graph in \autoref{fig:sample-rdf-graph}
and the following SPARQL \texttt{ASK} query:\\

\texttt{PREFIX foaf: <http://xmlns.com/foaf/0.1/>\\
\indent ASK \{ ?x foaf:name  "Alice" \}}\\

\noindent This query creates the following response
(as there is no person named Alice in the graph,
but only a~person named John X. Foobar):\\

\texttt{no}

\subsubsection{CONSTRUCT}

The \texttt{CONSTRUCT} query variant is used to extract
information from the SPARQL endpoint
and transform the results into valid RDF
specified by a~graph template.
The result is an RDF graph formed by taking
each query solution in the solution sequence,
substituting for the variables in the graph template,
and combining the triples into
a~single RDF graph by set union.
An example query is given below:\\

\noindent Given the following RDF graph:\\

\texttt{@prefix foaf: <http://xmlns.com/foaf/0.1/> .}\\
\texttt{\indent \_:a foaf:name "Alice" .}\\
\texttt{\indent \_:a foaf:mbox <mailto:alice@example.org> .}\\

\noindent Given the following SPARQL \texttt{CONSTRUCT} query:\\

\texttt{PREFIX foaf: <http://xmlns.com/foaf/0.1/>}\\
\texttt{\indent PREFIX vcard: <http://www.w3.org/2001/vcard-rdf/3.0\#>}\\
\texttt{\indent CONSTRUCT \{ <http://example.org/person\#Alice> vcard:FN ?name \}}\\
\texttt{\indent WHERE { ?x foaf:name ?name }}\\

\noindent This query creates the following
\texttt{vcard}~\cite{dawson1998vcard}
properties from the FOAF information:\\

\texttt{@prefix vcard: <http://www.w3.org/2001/vcard-rdf/3.0\#> .}\\ 
\texttt{\indent <http://example.org/person\#Alice> vcard:FN "Alice" .}

\section{Linked Data}

Linked Data~\cite{bernerslee2006linkeddata}
defines a~set of agreed-on best practices and
principles for interconnecting and publishing
structured data on the Web.
It uses Web technologies like HTTP~\cite{fielding1999http}
and URIs~\cite{bernerslee2005uri}
to create typed links between different sources.
The portal \url{http://linkeddata.org/}
defines\footnote{Accessed: November 19, 2012}
Linked Data as being
\textit{``about using the Web to connect related data that
wasn’t previously linked, or using the Web
to lower the barriers to linking data
currently linked using other methods''}.

\subsection{The Linked Data Principles}
\label{sec:linked-data-principles}

Tim Berners-Lee defines the four rules for Linked Data in a~W3C Design Issue~\cite{bernerslee2006linkeddata} as follows:

\begin{enumerate}
  \item Use URIs as names for things.
  \item Use HTTP URIs so that people can look up those names.
  \item When someone looks up a~URI, provide useful information,
        using the standards (RDF*, SPARQL).
  \item Include links to other URIs,
        so that they can discover more things.
\end{enumerate}

Linked Data uses RDF~\cite{klyne2004rdf} to create
typed links between things in the world,
the result is oftentimes referred to as the Web of Data.
As outlined before, RDF encodes statements
about things in the form of
\texttt{\{subject, predicate, object\}} triples.
If subject and object have URIs from different namespaces,
Bizer \emph{et al.} speak of \emph{RDF links}
in~\cite{heath2011linkeddata}.
An exemplary RDF link adapted from~\cite{bizer2009linkeddatastory}
stating that a~description of the movie Pulp Fiction
from the Linked Movie Database~\cite{hassanzadeh2009linkedmovie}
and a~description from DBpedia~\cite{auer2007dbpedia}
are indeed talking about the same movie
can be seen in \autoref{code:rdflink}.

\begin{lstlisting}[caption={[Exemplary RDF link.]{Exemplary RDF
  link stating that a~description of the movie Pulp Fiction from
  the Linked Movie Database~\cite{hassanzadeh2009linkedmovie}
  and a~description from DBpedia are indeed talking
  about the same movie.}},
  label={code:rdflink}]
http://data.linkedmdb.org/resource/film/77
http://www.w3.org/2002/07/owl#sameAs
http://dbpedia.org/page/Pulp_Fiction
\end{lstlisting}

\subsection{The Linking Open Data Cloud Diagram}
\label{sec:lodcloud}

The Linking Open Data (LOD) cloud
diagram~\cite{cyganiak2011lodcloud} is a~visualization effort
that shows datasets that have been published in
Linked Data~\cite{bernerslee2006linkeddata}
format by contributors to the Linking Open Data community project
and other individuals and organizations.
The objective is to identify existing datasets with open licenses,
convert them to RDF whilst obeying the Linked Data principles,
and finally publish them on the Web.
Due to its open structure, everyone can contribute to the project by publishing a~dataset and
interlinking it to existing datasets.
Today, the project includes datasets of major organizations
such as the BBC, Thomson Reuters, or the Library of Congress
to name just a~few.
The state of the LOD cloud has been examined
in~\cite{bizer2011statelodcloud}.
The latest LOD cloud diagram as of September 2011 can be seen in \autoref{fig:lod-cloud}.

\begin{figure}[htbp!]
\begin{center}
  \includegraphics[width=1.0\textwidth]{lod-cloud.png}    
  \caption[The Linked Open Data cloud as of September 2010.]
  {The Linked Open Data cloud as of September 2011, by Richard
   Cyganiak and Anja Jentzsch.
   Source: \url{http://lod-cloud.net/}.}    
  \label{fig:lod-cloud}
  \end{center}  
\end{figure}

\section{Conclusion}
In this chapter, we have first introduced the Semantic Web,
and compared it to the non-semantic Web.
We have shown how structured data in form of tables
is a~first step towards richer semantics.
An example of converting structured data from Wikipedia
to machine-readable data is the knowledge base DBpedia.
Further, we have looked at the intrinsic semantics
of HTML in versions 4 and 5,
and how through additional attributes
even richer semantics can be added
by the annotation formats Microformats and Microdata.
We have introduced the Resource Description Format (RDF)
and its different serializations.
On top of RDF, we have detailed
how the Semantic Web query language SPARQL
can be used to express queries across data sources.
Finally, we have shown how data on the Web
can be exposed as so-called Linked Data,
an effort which gets visible in the Linking Open Data cloud.
With these Semantic Web background technologies,
we have set the foundations for the coming chapters
that build upon those pillars.

\section*{Chapter Notes}
This chapter is partly based on the following publications:
\todo{Add publications}