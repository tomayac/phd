
% this file is called up by thesis.tex
% content in this file will be fed into the main document

%: ----------------------- introduction file header -----------------------
\chapter{Introduction}

% the code below specifies where the figures are stored
\ifpdf
    \graphicspath{{1_introduction/figures/PNG/}{1_introduction/figures/PDF/}{1_introduction/figures/}}
\else
    \graphicspath{{1_introduction/figures/EPS/}{1_introduction/figures/}}
\fi

% ----------------------------------------------------------------------
%: ------------------------------- content ----------------------------- 
% ----------------------------------------------------------------------

\section{Motivation and Problem Statement}

\section{Approach}

\section{The Web and Semantics}
Tim Berners Lee, inventor of the World-Wide Web (W3, WWW), or simply, the \emph{Web}, \emph{et al.}
write in~\cite{BernersLee1994}: \textit{``The World-Wide Web was developed
to be a pool of human knowledge, which would allow collaborators
in remote sites to share their ideas
and all aspects of a common project''}.
Since the earliest days at CERN,
the European Particle Physics Laboratory in Geneva, Switzerland,
the Web has scaled to a truly global system of interlinked hypertext documents
accessed via the Internet.

\emph{Semantics} is the study of meaning.
It focuses on the relation between signifiers, such as words, phrases, signs, and symbols,
and what they stand for, \emph{i.e.}, their denotata.
Michel Bréal is often identified as a~founder of modern semantics with his 1897
\emph{Essai de sémantique}~\cite{Breal1897}.
The \emph{Semantic Web} brings these two worlds together.

\section{Boundaries of this Work}

\section{Contributions}

\section{Thesis Structure}