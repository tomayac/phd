
% this file is called up by thesis.tex
% content in this file will be fed into the main document

%: ----------------------- introduction file header -----------------------
\chapter{Introduction}

% the code below specifies where the figures are stored
\ifpdf
    \graphicspath{{1_introduction/figures/PNG/}{1_introduction/figures/PDF/}{1_introduction/figures/}}
\else
    \graphicspath{{1_introduction/figures/EPS/}{1_introduction/figures/}}
\fi

% ----------------------------------------------------------------------
%: ------------------------------- content ----------------------------- 
% ----------------------------------------------------------------------

\section{Motivation and Problem Statement}
A~very open definition of the word \emph{event}
given by WordNet~\cite{Princeton:WordNet} is
\emph{something that happens at a given place and time}.
We are indeed surrounded by events, most of them are of little to no interest for us.
A~concert somewhere in the world of a~band that we do not like may be a~good example.
For some events, however, we may care more, for example,
a~concert of a~band that we do like, even if it takes place at a far away location.
Finally, very few events, we actually may attend,
like a~concert of our favorite if it takes place in our city.

A~well-known German saying goes that one cannot dance at two weddings at the same time.
If two events we are interested in take place at overlapping times,
we need to take a decision to attend the one, and to skip the other.

All this motivates the need for \emph{event summarization}.
If there is an event that we could not attend, but that we are interested in,
a~good event summarization can help us get a~feeling for the event's atmosphere.
Similarly, if there is an event that we attended,
we can revive the event's most fascinating moments based on the event summarization.

Event summarization covers textual, as well as multimedia content.
For bigger events, official TV or newspaper journalists report on site,
and oftentimes the event organizers themselves share official media material,
or even an official press package with event reports and images.

We argue that by leveraging content that gets shared on social networks,
a~more \emph{authentic}, \emph{concise}, \emph{divers}, and also more \emph{interesting}
view on events is possible than by limiting oneself to official media content.
We validate these subjective criteria with experiments for events of different categories
such as sports, politics, culture, music, conferences, etc.
In addition to the subjective criteria mentioned before,
we also compare objective criteria such as \emph{creation time} and \emph{efficiency}
of our approach to event summary generation with officially produced summaries.

\todo{media galleries definition}

\todo{define: authentic, diverse, etc.}

\todo{1.1 some duplicate content in 1.2}

\section{Research Question and Hypothesis}
Our leading research question for this thesis can be formulated as follows:
 
\textit{``Can user-customizable media galleries that summarize given events
be created just based on textual and multimedia data from social networks''?}

\noindent Our hypothesis that we will test in this thesis is the following:

\textit{User-customizable media galleries that summarize given events and
that are created solely based on textual and multimedia data from social networks
provide a~subjectively more \emph{authentic}, \emph{concise}, \emph{divers},
and more \emph{interesting} view on events than media galleries that summarize events
and that are created solely based on official media content,
and do so in an objectively \emph{shorter} creation time and more \emph{efficiently}.}

\section{Approach}
\todo{add diagram from thesis proposal}

\section{Contributions}
\todo{add contributions as prose, "bragging rights"}

\section{The Web and Semantics}
Tim Berners Lee, inventor of the World-Wide Web (W3, WWW), or simply, the \emph{Web}, \emph{et al.}
write in~\cite{BernersLee1994}: \textit{``The World-Wide Web was developed
to be a pool of human knowledge, which would allow collaborators
in remote sites to share their ideas
and all aspects of a common project''}.
Since the earliest days at CERN,
the European Particle Physics Laboratory in Geneva, Switzerland,
the Web has scaled to a truly global system of interlinked hypertext documents
accessed via the Internet.

\emph{Semantics} is the study of meaning.
It focuses on the relation between signifiers, such as words, phrases, signs, and symbols,
and what they stand for, \emph{i.e.}, their denotata.
Michel Bréal is often identified as a~founder of modern semantics with his 1897
\emph{Essai de sémantique}~\cite{Breal1897}.
The \emph{Semantic Web} brings these two worlds together.

\section{Boundaries of this Work}

\section{Contributions}

\section{Thesis Structure}