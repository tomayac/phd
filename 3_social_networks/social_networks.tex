\chapter{Social Networks}
\label{cha:social-networks}

% the code below specifies where the figures are stored
\ifpdf
    \graphicspath{{3_social_networks/figures/PNG/}{3_social_networks/figures/PDF/}{3_social_networks/figures/}}
\else
    \graphicspath{{3_social_networks/figures/EPS/}{3_social_networks/figures/}}
\fi

From the first ever email to video calls on the go,
the Internet has always been about communication.
Historically, communities formed around Usenet mailing lists or Bulletin Board Systems, starting from the early eighties,
often around all sorts of topics like fine arts,
literature, and philosophy (\emph{e.g.}, \texttt{humanities.classics}
or \texttt{humanities.\-design.misc}).
Then, starting from the late eighties, Internet Relay Chats (IRC)
allowed people to communicate interactively and in real-time,
organized in channels (\emph{e.g.}, \texttt{\#linux}).
Starting from the nineties, blogs began to spread,
reaching mainstream popularity somewhere in mid-2000.
While the early social communities
where created entirely \emph{ad-hoc}
whenever someone logged in to a~system,
the first social networks,
among them \url{http://sixdegrees.com/} in 1997,
allowed people to maintain a~public profile
with a~list of connections (\emph{friends})
that others could browse.
In~\cite{boyd2007socialnetworksites}, boyd
(\emph{sic}\footnote{http://www.danah.org/name.html})
and Ellison define the term
\emph{social network site (SNS)} as follows.

\begin{quotation}
  \textit{``We define social network sites as web-based services
  that allow individuals to
  (1) construct a~public or
  semi-public profile within a~bounded system,
  (2) articulate a~list of other users
  with whom they share a~connection, and
  (3) view and traverse their list of connections
  and those made by others within the system.
  The nature and nomenclature of these connections
  may vary from site to site.''}
\end{quotation}

Literature on social networks oftentimes uses the term SNS,
however, in order to differentiate ourselves
from the therein defined idea of social network,
we decided to avoid the term altogether in favor of a~more open
definition of social network,
which we detail in the following. 

\section{Definition of Terms Used in this Thesis}
\label{sec:definition}

In this section, we define the terms
that we will use throughout this thesis
in order to avoid any ambiguity.
In particular, we highlight that social networks have
different levels of support for media items.

\begin{description}
  \item[Social Network:]
       A~social network is an online service or media platform
       that focuses on building and reflecting
       relationships among people
       who share interests and/or activities.
  \item[Media Item:]
       A~media item is defined as
       a~photo\footnote{We chose the term \emph{photo}
       over the term \emph{image} as 
       Facebook, Twitter, and \googleplus use it}
       or video
       file that gets distributed via a~social network.
  \item[Micropost:]
       A~micropost is defined as a~textual status message
       that can optionally be accompanied by a~media item.
  \item[Hashtag] The \texttt{\#} symbol, called a~hashtag,
       is used to mark keywords or topics in a~micropost.
       It was created organically by Twitter users
       as a~way to categorize messages.
       People use the hashtag symbol \texttt{\#} before a~relevant keyword
       or phrase (no spaces) in microposts to categorize them
       and help them show more easily in
       search.\footnote{Adapted from
       \url{https://support.twitter.com/articles/49309-what-are-hashtags-symbols},
       accessed December 7, 2012.}
\end{description}

The boundary between \emph{social networks} and
\emph{media platforms} is blurred.
Several media sharing platforms, \emph{e.g.},
YouTube,\footnote{http://youtube.com/}
enable people to upload content
and optionally allow other people to react
to this content in the form of comments, likes or dislikes.
On other social networks, \emph{e.g.},
Facebook,\footnote{http://facebook.com/}
users can update their status, post links to stories,
upload media content, and also give readers the option to react.
Finally, there are hybrid clients, \emph{e.g.},
TweetDeck\footnote{http://www.tweetdeck.com/}
by Twitter together with
Twitpic\footnote{http://twitpic.com/},
where social networks integrate with media platforms,
typically via third-party applications.

\section{Description of Popular Social Networks}
\label{sec:description-of-popular-social-networks}

In this section, we introduce several social networks
and some of their key features
that are relevant for our approach.
As we treat all networks the same---%
independent from their not always publicly known user population---%
they are listed in alphabetic order.
For active participation,
all social networks require users to be logged in.
In the description below, we thus assume a~logged in user.

\subsection{Facebook}

Facebook (\url{http://www.facebook.com/})
is a~social networking service launched in February 2004,
operated and owned by the American multinational
Internet corporation Facebook, Inc.
At time of writing, Facebook is the most popular social network
with one billion monthly active
users\footnote{\url{http://newsroom.fb.com/Key-Facts},
accessed November 19, 2012}
as of October 2012.
Facebook has native photo and video support,
allowing people to upload an unlimited amount of media items.
Photos and videos can also be recorded \emph{ad hoc} via webcam.
People can \emph{Like} content via a~designated Like button
that can also be embedded on other websites.
Initially, the button was called the \emph{Awesome} button,
but eventually\footnote{\url{http://www.quora.com/Facebook-Inc-company/Whats-the-history-of-the-Awesome-Button-that-eventually-became-the-Like-button-on-Facebook}}
got rebranded to its current form.
Individual microposts can also be shared.
Facebook has a~bidirectional relationship model (friend model),
with an optional unidirectional relationship model (follow model),
typically for following celebrities, remote friends, \emph{etc.}

\subsection{Flickr}

Flickr (\url{http://www.flickr.com/})
is a~photo and video hosting online community
created by Ludicorp in 2004 and acquired by Yahoo! in 2005.
Users can upload a~limited or unlimited number of photos or videos
to the service, depending on their account type (Free or Pro).
People can \emph{Favorite} photos they like
via a~designated Favorite button.
Flickr has a~unidirectional relationship model (follow model),
however, also allows people to mark other users as friends
or family \emph{without} the other party having to confirm.
Following an urgent plea from Flickr
users,\footnote{\url{http://dearmarissamayer.com/}}
where they complained that Yahoo!
had semi-abandoned the service for too long,
Flickr is now said to be revived under new CEO Marissa
Mayer.\footnote{\url{http://www.flickr.com/dearinternet}}

\subsection{\googleplus}

\googleplus (\url{http://google.com/+}),
sometimes transcribed as Google Plus
and abbreviated as \gplus, is Google's social network.
It was opened to the general public on September 20, 2011.
\googleplus has native photo support.
Photos can either be manually uploaded
when authoring a~new micropost,
or be automatically uploaded via the \googleplus
mobile application.
External videos, for example, from
the also Google-owned online video platform YouTube,
but also from other services,
get displayed in an inline view
so that they can be viewed directly on the website.
However, the network also allows for
videos to be uploaded directly,
or to be recorded \emph{ad hoc} via webcam.
People can \emph{\plusone}
(pronounced like a~verb ``to plus-one'') content they like
via a~designated \plusone button
that can also be embedded on other websites.
Individual microposts can also be shared.
\googleplus has a~unidirectional relationship model
(follow model).

\subsection{Img.ly}

Img.ly (\url{http://img.ly/})
is a~photo hosting service operated by 9elements GmbH
and founded in 2009.
It integrates deeply with Twitter, however,
can also be used independently.
Img.ly integrates with Twitter's \emph{Tweet} button. 
The service has no own relationship model,
but uses a~user's social graph on Twitter.

\subsection{Imgur}

Imgur (\url{http://imgur.com/})
is a~photo hosting service
founded by Alan Schaaf in February 2009.
While the service is deeply integrated with Twitter
and Facebook, it can be used independently as well.
Imgur integrates with all major social networks,
and also has designated \emph{Like} and \emph{Dislike} buttons.
The service has no own relationship model,
but uses a~user's social graph on Facebook.

\subsection{Instagram}
\label{sec:instagram}

Instagram (\url{http://instagram.com/})
is a~mobile photo sharing application
that was acquired by Facebook in April 2012.
The application allows users to apply filters to photos.
These photos can then be shared on external social networks
like Facebook, Twitter, or \googleplus,
and are also visible on Instagram's own social network.
The service launched in October 2010.
Instagram has native photo support, but,
does not support videos.
People can \emph{Like} content via a~designated Like button
from within the Instagram application.
Instagram has a~unidirectional relationship model (follow model).

\subsection{Lockerz}

Lockerz (\url{http://lockerz.com/}) is an international
social commerce website based in Seattle, WA. 
In 2011, Lockerz acquired the photo sharing service Plixi,
whih was formerly known as TweetPhoto.
Lockerz keeps Plixi's service as a~media platform running
under the new Lockerz branding.
While the service is deeply integrated with Twitter,
it can be used independently as well.
People can \emph{Love} content they like via
a~designated Love button,
but the service is also integrated with all major social networks.

\subsection{MobyPicture}

MobyPicture (\url{http://www.mobypicture.com/})
is a~mobile messaging service
owned by entrepreneur Mathys van Abbe.
Users of the service can upload an unlimited number of
photos and videos to the service.
MobyPicture integrates with a~number of
third-party social networks.
The service natively supports videos and photos,
which can either be uploaded, or be recorded \emph{ad hoc}
via webcam.
People can \emph{Favorite} content they like via
a~designated Favorite button,
however, the service also integrates with Google's
\emph{\plusone} button and Twitter's \emph{Tweet} button. 
MobyPicture has a~unidirectional relationship model
(follower model).

\subsection{Myspace}

Myspace (\url{http://www.myspace.com/}),
formerly MySpace and My\_\_\_\_\_ (\emph{sic}), is
a~social networking service owned by Specific Media LLC
and pop star Justin Timberlake.
The social network launched in August 2003.
Once the most visited website
in the United States in June 2006,
the network's importance is steadily declining since.
Instead of as a~social networking website,
Myspace has attempted to redefine itself
as a~social entertainment website,
putting more focus on music, movies, celebrities, and TV.
At time of writing, yet another redefinition process is
ongoing.\footnote{\url{https://new.myspace.com/}}
As such, Myspace has native photo, video, and,
via special musician profiles, audio support.
Videos can either be uploaded,
or be recorded \emph{ad hoc} via webcam.
In January 2012, a~rebranding strategy to Myspace TV
in collaboration with Panasonic was unveiled
with an exclusive focus on social TV that would allow people
to watch and comment on videos.
People can \emph{Like} certain content via a~designated Like link.
Myspace has a~bidirectional relationship model (friend model),
with an optional unidirectional relationship model (follow model),
typically meant for following celebrities like artists,
musicians, \emph{etc.}

\subsection{Photobucket}

Photobucket (\url{http://photobucket.com/})
is a~photo and video hosting service
founded in 2003 by Alex Welch and Darren Crystal.
It was acquired by Fox Interactive Media in 2007.
In June 2011, Twitter announced an exclusive partnership
with Photobucket that made the service
the default photo sharing platform for Twitter,
used for its native media item support.

\subsection{Twitpic}

Twitpic (\url{http://twitpic.com/})
is a~service that allows users to upload photos and videos.
It optionally integrates with Twitter.
Twitpic was launched in 2008 by Noah Everett.
While Twitpic can be used independently from Twitter,
the integration is made easy with Twitpic usernames and passwords
being the same as the ones on Twitter.
Twitpic integrates with Twitter via the \emph{Tweet} button.
The service has no own relationship model,
but uses a~user's social graph on Twitter.

\subsection{Twitter}
\label{sec:twitter}

Twitter (\url{http://twitter.com/})
is an online social networking service
and microblogging service
that enables its users to send and read microposts
of up to 140 characters.
These microposts are referred to as \emph{tweets}.
Twitter was founded in March 2006 by Jack Dorsey
and launched to the public in July 2006.
The website is ranked among the top-10 websites globally
by the Web information company
Alexa.\footnote{\url{http://www.alexa.com/topsites},
accessed November 19, 2012}
As of August 2011, Twitter has native photo support,
which allows users to upload photos to the service.
However, at time of writing, it is not possible to 
record photos or videos \emph{ad hoc} via webcam.
Videos are not supported natively, however,
likewise the situation with photos before
(and also in part still today),
an ecosystem of media platforms takes care of
hosting media items on behalf of Twitter users.
These third-party-hosted media items
can be linked from within tweets.
People can \emph{ReTweet} content they like either
via a~designated ReTweet button,
or---following the prior, but still widely popular,
manual ReTweet convention---by
quoting a~Twitter user by prepending ``RT @username:''
in front of the original tweet.
In addition to that, Twitter offers
a~\emph{Tweet} button that can be embedded on other websites.
Twitter has a~unidirectional relationship model (follow model).

\subsection{Yfrog}

Yfrog (\url{http://yfrog.com/})
is a~photo and video hosting service run by ImageShack
that was launched in February 2009.
While the service is deeply integrated with Twitter,
it can be used independently as well.
Yfrog integrates with Twitter's \emph{Tweet} button.
The service has no own relationship model,
but uses a~user's social graph on Twitter.

\subsection{YouTube}

YouTube (\url{http://www.youtube.com/})
is a~video sharing website founded in February 2005.
In November 2006, YouTube was acquired by Google,
and now operates as a~subsidiary of the company.
It allows people to upload, view,
and share an unlimited number of videos.
YouTube has native video support, but, does not support photos.
Videos can be uploaded, or be recorded \emph{ad hoc} via webcam.
People can \emph{Like} or \emph{Dislike} content
via designated Like or Dislike buttons.
YouTube has a~unidirectional relationship model (follow model).

\section{Decentralized Social Networks}
All social networks presented up to now are centralized networks.
In contrast, \emph{distributed}, or also referred to as
\emph{decentralized social networks}, are
social network services that are decentralized and distributed
across different providers, with a~special focus on
portability, interoperability, and federation capability,
\emph{i.e.}, an agreement upon standards of operation
in a~collective fashion.
Decentralized, protocol-based systems
offer users a~choice of providers, which implies
that, if one provider should terminate their service,
the user is free to take out her data and start
where she left off with a~different provider.
As a~final advantage, governments cannot effectively censor
decentralized social networks,
as this would be impracticable due to the distributedness.

\subsection{Examples of Decentralized Social Networks}

In the following, we will list representative efforts 
in the direction of truly decentralized social networks.
This list is not meant to be complete,
but covers the efforts that received the most media attention
in the years 2011 and 2012.

\subsubsection{StatusNet}

A~first example of decentralized social network software providers
is StatusNet ({\url{http://status.net/}),
which provides an open-source implementation of the
OStatus\footnote{\url{http://gitorious.org/projects/ostatus/}}
open standard, most prominently deployed
at \url{http://identi.ca/}.

\subsubsection{The DIASPORA* Project}

A~second example is the DIASPORA* project (\url{http://diasporaproject.org/}),
which provides a~free and open-source personal Web server component
referred to as \emph{pod} that allows
participants in the project to form nodes
that span the distributed Diaspora social network.

\subsubsection{Tent}

Third, there is Tent\texttrademark~(\url{https://tent.io/}).
Tent is an open-source protocol for distributed social networking
and personal data storage.
Anyone can run a~Tent server,
or write an app or alternative server implementation
that uses the Tent protocol.
Users can take their content and relationships with them
when they change or move servers.
Tent supports extensible data types,
so developers can create new kinds of interactions.
Rather than running an own server,
users can also rely on Tent.is (\url{https://tent.is/}),
a~service which hosts Tent servers
and basic applications for users.
Yet the global site feed\footnote{\url{https://app.tent.is/global}},
suggests that the service is not very active.

\subsection{Current Status of Decentralized Social Networks}

None of the decentralized social networks could reach
a~critical mass of users and/or network activity as of yet.
We will therefore not consider them for this thesis.

\section{Classification of Social Networks}
\label{sec:classification-of-social-networks}

As motivated in \autoref{sec:definition},
different social networks have varying support
for media items, ranging from native support
in media-centric social networks,
to optional support in micropost-centric social networks.
In order to differentiate social networks by their
media item support level,
we introduce a~classification of social networks as follows.

\begin{itemize}
  \item \emph{First-order support}:
        The social network is centered around media items
        and posting requires the inclusion of a~media item
        (\emph{e.g.}, YouTube, Flickr).
  \item \emph{Second-order support}:
        The social network lets users upload media items,
        but it is also possible to post purely textual messages
        (\emph{e.g.}, Facebook).
  \item \emph{Third-order support}:
        The social network has no direct support for media items,
        but relies on third-party media platforms
        to host media items, which are linked to the status update
        (\emph{e.g.}, Twitter relying on third-party video hosting via Twitpic).
\end{itemize}

In this chapter, we consider 11 different social networks
that represent all together most of the market share.
The criteria for inclusion follow
a~study~\cite{levine2011howpeopleshare}
performed by the company Sysomos, specialized in social media
monitoring and analytics.
\autoref{tab:platforms} lists these social networks according to the categorization defined above.

\begin{sidewaystable}
  {\small
  \begin{tabular}{|l|l|l|p{8cm}|}
    \hline
    \textbf{Social Network} & \textbf{URL} & \textbf{Category} & \textbf{Comment}\\
    \hline
	\googleplus & \url{http://google.com/+} & second-order & Media item links are returned via the \googleplus API.\\
	Myspace & \url{http://myspace.com} & second-order & Media item links are returned via the Myspace API.\\
	Facebook & \url{http://facebook.com} & second-order & Media item links are returned via the Facebook API.\\
	Twitter & \url{http://twitter.com} & second-/third-order & In second order mode, media item links are returned via the Twitter API. In third order mode, Web scraping or media platform API usage are necessary to retrieve media item links. Many people still use Twitter in third order mode.\\\hline
	Instagram & \url{http://instagram.com} & first-order & Media item links are returned via the Instagram API.\\
	YouTube & \url{http://youtube.com} & first-order & Media item links are returned via the YouTube API.\\
	Flickr & \url{http://flickr.com} & first-order & Media item links are returned via the Flickr API.\\
	MobyPicture & \url{http://mobypicture.com} & first-order & Media platform for Twitter. Media item links are returned via the MobyPicture API.\\
	Twitpic & \url{http://twitpic.com} & first-order & Media platform for Twitter. Media item links are returned via the Twitpic API.\\
	Img.ly & \url{http://img.ly} & first-order & Media platform for Twitter. Media item link must be retrieved via Web scraping.\\
	Yfrog & \url{http://yfrog.com} & first-order & Media platform for Twitter. Media item links must be retrieved via Web scraping.\\
	\hline
  \end{tabular}
  }
  \caption[11 social networks with different level of support for media items]
  {11 social networks with different level of support for media
   items and techniques needed to retrieve them. \todo{Check final orientation}}
  \label{tab:platforms}  
\end{sidewaystable}

\section{Conclusions}
Alongside the Semantic Web background technologies
that were introduced in the previous chapter,
social networking sites form the backbone of this thesis.
In this chapter, we have thus first defined the terms of
\emph{social network}, \emph{micropost}, \emph{media platform},
and \emph{media item}.
Subsequently, we have introduced and described in detail
the most popular social networking sites and media platforms.
Different social networking sites have
a~different level of support for media items.
We have therefore classified the social networking site
landscape accordingly.
In the upcoming chapters, we will get to the core of
micropost annotation, media item extraction from microposts,
followed by media item deduplication, clustering, and ranking.
Finally, we will close the core part with media item compilation.

\section*{Chapter Notes}
This chapter is partly based on the following publications:
\todo{Add publications}

\bibliographystyle{plainnat}
\bibliography{backmatter/references}
