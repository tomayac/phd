
% Thesis Abstract -----------------------------------------------------


%\begin{abstractslong}    %uncommenting this line, gives a different abstract heading
\begin{abstracts}        %this creates the heading for the abstract page
Mobile devices like smartphones and digital cameras together with social networks enable people to generate,
share, and consume enormous amounts of media items, both \textit{en route} or at home.
Common search operations, for example searching for a single music video based on artist
and title on video platforms such as YouTube,
can be achieved both based on potentially shallow human-generated metadata,
or based on more profound content analysis,
driven by Optical Character Recognition (OCR) or Automatic Speech Recognition (ASR).
However, more advanced use cases, such as \emph{summaries} or \emph{compilations} of \emph{several} media items
covering a certain event, are hard, if not impossible to fulfill at large scale.
Example events can be a keynote speech at a conference,
a music concert in a stadium, or a natural catastrophe in a country.
At such events, given a stable network connection,
media items are published on social networks as the event happens.
The main research question can thus be formulated as\\

\noindent \textit{``Can user-customizable media galleries that summarize given events
be\linebreak created solely based on textual and multimedia data from social networks?''}\\

In this thesis, we develop a novel interactive Web application for media item enrichment,
leveraging social networks, utilizing the Web of Data,
Content-based Image and Video Retrieval (CBIR, CBVR) techniques,
and fine-grained media item addressing schemes like Media Fragments URIs,
to provide a scalable and near realtime solution to realize the above use cases.
We detail and evaluate our approach:
first, we recognize and disambiguate named entities in relevant microposts
from social networks that contain links to media items, or to pages that host media items.
Second, we extract the raw media item data from social networks and relate them to the originating micropost.
Third, we deduplicate media items using CBIR and CBVR techniques.
Fourth, we rank the deduplicated list of media items according to specified ranking criteria,
and finally compile the top-$n$ ranked media items to a user-customizable media gallery.

\end{abstracts}
%\end{abstractlongs}

% ---------------------------------------------------------------------- 
