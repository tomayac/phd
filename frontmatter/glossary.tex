% this file is called up by thesis.tex
% content in this file will be fed into the main document

% Glossary entries are defined with the command \nomenclature{1}{2}
% 1 = Entry name, e.g. abbreviation; 2 = Explanation
% You can place all explanations in this separate file or declare them in the middle of the text. Either way they will be collected in the glossary.

% required to print nomenclature name to page header
\markboth{\MakeUppercase{\nomname}}{\MakeUppercase{\nomname}}

\nomenclature{NLP}{Natural Language Processing}
\nomenclature{CBIR}{Content-based Image Retrieval} 
\nomenclature{CBVR}{Content-based Video Retrieval} 
\nomenclature{RDF}{Resource Description Framework}
\nomenclature{W3}{World-Wide Web}
\nomenclature{WWW}{World-Wide Web}
\nomenclature{CERN}{European Organization for Nuclear Research}
\nomenclature{HTML}{Hypertext Markup Language}
\nomenclature{URI}{Unique Resource Identifier}
\nomenclature{JSON}{JavaScript Object Notation}
\nomenclature{Turtle}{Terse RDF Triple Language}
\nomenclature{CURIE}{Compact URI}
\nomenclature{SPARQL}{SPARQL Protocol and RDF Query Language}
\nomenclature{W3C}{World Wide Web Consortium}
\nomenclature{LOD}{Linking Open Data}
\nomenclature{SNS}{Social Network(ing) Site}